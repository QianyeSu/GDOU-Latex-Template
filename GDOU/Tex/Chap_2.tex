\chapter{广东海洋大学研究生学位论文格式及书写规范}
学位论文是研究生培养质量和学术水平的集中体现。高质量、高水平的学位论文不仅在内容上有创造性和创新性,而且在表达方式上具有一定的规范性和严谨性。为了统一学位论文的撰写和编辑的格式,便利信息系统的收集、存储、处理、加工、检索,根据中华人民共和国标准《科学技术报告、学位论文、学术论文的编写格式》(GB7713-87)及《文后参考文献著录规则》(GB/T7714-2005)的要求,结合我校实际,特制订本规范。
\section{基本要求}
本规范适用于向我校申请博士、硕士学位的各类研究生学位论文。各研究生培养单位应对研究生学位论文进行规范性审查,凡不符合本规范的学位论文,一律不予受理学位申请。
\section{语言要求}
根据《中华人民共和国国家通用语言文字法》的有关规定,研究生学位论文一般用中文(简体汉字)撰写。

根据《国务院学位委员会关于普通高等学校授予来华留学生我国学位试行办法》(学位〔1991〕14号)的有关规定,攻读我国哲学、经济学、法学、教育学、文学、历史学以及艺术等学科、专业的博士和硕士学位的来华留学生,应采用汉语撰写和答辩论文;攻读其他学科、专业的博士和硕士学位的来华留学生,其学位论文可以用汉语、英语和法语撰写及答辩,但必须附5000字的详细中文摘要。
\section{字数要求}
自然科学类硕士学位论文字数一般不少于2万字,人文社会科学类硕士学位论文不少于3万字,艺术硕士学位论文字数不少于0.5万字;自然科学类博士学位论文字数一般不少于5万字,人文社会科学类博士学位论文字数不少于8万字(不含中英文摘要、参考文献、附录和致谢等部分)。
\section{论文内容规范要求}
论文内容应层次分明、数据可靠、文字简练、说明透彻、推理严谨、立论正确。论文内容一般应由以下主要部分组成:1.封面;2.学位论文答辩委员会组成;3.学位论文原创性声明和版权使用授权书;4.中文摘要;5.英文摘要;6.目录;7.主要符号说明(必要时);8.论文正文;9.参考文献;10.附录(必要时);11.致谢;12.作者简介;13.导师简介。

各部分具体要求如下

\subsection{封面}
封面的内容包括学位论文分类号、密级、UDC、学号、学位论文题目、作者姓名、指导教师、学位类别、专业(领域)、研究方向、培养单位、日期等,具体见样例1-3。

1.题目。学位论文题目应能概括论文的主要内容,切题、简洁,一般不宜太长,要求在30字以内。

2.指导教师。第一指导教师一定是通过我校学位评定委员会批准的指导教师。专业学位第二导师为校外导师。

3.学位类别。我国学位类别分为学术型学位与专业型学位。学术型学位按照学科门类授予,分别为哲学、经济学、法学、教育学、文学、历史学、理学、工学、农学、医学、军事学、管理学、艺术硕士学位/博士学位。专业型学位按照专业学位类型授予。

4.专业(领域)、研究方向。专业(领域)按《授予博士、硕士学位和培养研究生的学科、专业目录》(1997)中规定的学科、专业填写。研究方向(学术型):按照入学时确定的研究方向填写。

5.分类号。中图分类号(http://www.ztflh.com/)和UDC号(《国际十进位分类法》)可在图书馆查阅获得。

6.密级。经批准,确属涉密的论文,需注明密级及保密年限。

7.培养单位。各研究生培养学院全称。

8.日期。学位授予时间:上半年授予学位填写20XX年6月;下半年授予学位填写20XX年12月。

\subsection{学位论文答辩委员会组成}
此页需学位论文答辩委员会组成成员签字。指导教师为第一指导教师签字,时间按实际答辩时间填写,具体见样例4。
\subsection{学位论文原创性声明和版权使用授权书}
单独设一页,附于学位论文摘要之前,需研究生、指导教师本人签字,具体见样例5。
\subsection{中文摘要}
中文摘要应说明研究目的、研究方法、成果和结论,突出学位论文中的创造性成果和新见解,语言力求精练。硕士学位论文中文摘要字数一般为800-1000字,博士学位论文中文摘要字数一般为1000-1200字。

中文摘要格式具体见样例6,居中打印论文题目(小二号黑体),换行居中打印“摘要”(小三号黑体)及摘要内容(小四号宋体),每段开头空二格。摘要内容后换行顶格编排“关键词”(小四号黑体)。每篇论文应选取3-5个关键词,每一关键词之间用“;”分开。关键词应从《汉语主题词表》等词表中选用规范词。
\subsection{英文摘要}
英文摘要格式和内容与中文摘要相对应,另起一页,具体见样例7。居中打印论文英文题名(小二号Times New Roman字体,加粗),换行居中打印编排“Abstract”(小三号Times New Roman,加粗)及英文摘要内容(小四号Times New Roman字体),摘要内容每段开头留四个字符空格。摘要内容后下换行顶格编排“Keywords”。为了国际交流,应标注与中文对应的英文关键词。
\subsection{目录}
目录是论文组成部分的小标题。目录一般列至三级标题,逐项标明页码,具体见样例8。各级标题应简明扼要、点出各部分主要内容。学位论文的页码编排为:正文和后置部分用阿拉伯数字编连续码,前置部分用罗马数字单独编连续码(封面除外)。

“目录”两字居中(小三号黑体),下空一行为章、节、小节及其开始页码(靠右顶格)。章、节、小节分别以阶梯式排列:1、1.1、1.1.1依次标出。章、节、小节的题名与页码之间用“......”连接。除“目录”两字外,其余字体均为小四号宋体。

\subsection{符号说明(必要时)}
如果论文中使用了大量的物理量符号、标志、缩略词、专门计量单位、自定义名词和术语等,应编写成注释说明汇集表,说明论文中所用符号所表示的意义及单位(或量纲)。若上述符号和缩略词使用数量不多,可以不设专门的注释说明汇集表,仅在论文中出现时加以说明。“符号说明”四字居中(小三号黑体)。

\subsection{论文正文}
论文正文是主体,主体部分应从另页右页开始,每一章应另起页。一般由序号标题、文字叙述、图、表格和公式等五个部分构成。写作形式可因研究内容的性质不同而变化,一般可包括绪论(或综述)、理论分析、计算方法、实验装置和测试方法、实验结果分析和讨论、研究成果、结论及意义等。正文采用小四号宋体,行间距为20磅,文中英文、罗马字符一般采用Time New Roman正体,按规定应采用斜体的采用斜体。

1.标题
论文主体部分可根据需要划分为不同数量的章、节、小节,章、节、小节编号全部顶格排,编号与标题之间空1个字的间隙,详见样例9。

1ㄩ(一级标题,小三号黑体,单列1行,段前、段后间距为1行);

1.1ㄩ(二级标题,四号黑体,单列1行,段前、段后间距为0.5行);

1.1.1ㄩ(三级标题,小四号黑体,单列1行);

2.图、表、公式

文中的图、表、公式、附注等一律用阿拉伯数字按章及顺序编号,如图1-1,表2-2,公式(3-10)等。

图序及图名居中置于图的下方,图中的术语、符号、单位等应与正文表述所用一致。文中所列图形应有所选择,照片不得直接粘贴,须经扫描后以图片形式插入。

表序及表名置于表的上方,表中参数应标明量和单位的符号。表的编排应采用国际通用的三线表。续表均应重复表头,如表需转页接排,随后各页应重复表的编号,其后跟表题(可省略)和“续”置于表上方。

图序及图名、表序及表名采用五号宋体加粗。图表中文字采用五号(或小五号)宋体;图例说明、表注采用小五号宋体;图、表单倍行间距。若图或表中有附注,采用英文小写字母顺序编号,图例说明(或表注)用最简练的文字、横排于图名(表名)下方。

公式一般居中对齐,公式编号用小括号括起,右对齐,其间不加线条。

图、表、公式等与正文之间要有一定的行间距。
\subsection{参考文献}
文后参考文献只列出作者直接阅读过、在正文中被引用过的文献资料,务必实事求是。参考文献一律列在正文的末尾,不得放在各章之后。人文社科类学位论文中不宜用文后参考文献列出的注释可采用页下注,包括对学位论文中某些关键词句、论点的详细说明。在引用别人的科研成果时,应在引用处加以说明,遵循学术道德规范,严禁论文抄袭、剽窃等学术不端行为。

参考文献采用顺序编码制组织,即按中文引用的顺序将参考文献附于文末。作者姓名写到第三位,余者写“,等”或“,et al.”。“参考文献”四字居中(小三号黑体),空一行左起按顺序依次列出参考文献,将序号置于方括号内(如[1]),用小四号宋体字,所有符号均用半角编排,悬挂式排列。

可列于参考文献的文献类型包括期刊(J)、专著(M)、论文集(C)、学位论文(D)、标准文献(S)、技术报告(R)、专利(P)、电子文献(电子公告:EB,磁带:DK,磁盘:DK,光盘:CD,联机网络:OL)、数据库(DB)的文献等。其著录格式分别如下:

1.期刊文章(文献类型标志:J)

[序号]作者.题名[J].刊名,年,卷(期):起止页码.

2.专(译)著(文献类型标志:M)

[序号]专著作者.书名[M].(版次)(第1版不标注).出版地:出版者,出版年:起止页码.

3.论文集中析出文献(文献类型标志:C)

[序号]析出文献主要责任人.析出文献题名[C].论文集主要责任者.论文集题名.出版地:出版者,出版年:析出文献起止页码.

4.学位论文(文献类型标志:D)

[序号]学位论文作者.题名[D].保存地点:保存单位,年份.

5.专利文献(文献类型标志:P)

[序号]申请者或所有者.专利题名[P].国别:专利号,公告或公开日期.

6.标准文献(文献类型标志:S)

[序号]标准编号,标准名称[S].出版地:出版者,出版年.

7.技术报告(文献类型标志:R)

[序号]主要责任者.报告题名[R].报告地:报告会主办单位,年份.

8.报纸文章(文献类型标志:N)

[序号]主要责任者.题名[N].报纸名,出版日期(版次).

9.电子文献

[序号]主要责任者.题名[文献类型标志/文献载体标志].出版地:出版者,出版年(更新或修改日期)[引用日期].获取和访问路径.

示例:

[1]高景德,王祥珩.交流电机的多回路理论[J].清华大学学报(自然科学版),1987,27(1):1-8.

[2]Nadkarni M A,Nair C K K, Pandey V N,et al.Characterization of alpha-galactosidase from corynebacterium murisepticum of its induction[J].J Gen App Microbiol,1992,38(1):23-24.

[3]竺可桢.物理学[M].北京:科学出版社,1973:1-3.

[4]霍夫期塔.禽病学:下册[M].(第7版).胡祥壁.译.北京:农业出版社,1981:798-799.

[5]孙品一.高校学报编辑工作现代化特征[C].中国高等学校自然科学学报研究会.科技编辑学论文集(2).北京:北京师范大学出版社,1998:10-22.

[6]张和生.地质力学系统理论[D].太原:太原理工大学,1998.

[7]姜锡洲.一种温热外敷药制备方案[P].中国:881056078,1983-08-12.

[8]GB/T 16159—1996,汉语拼音正词法基本规则[S].北京:中国标准出版社,1996.

[9]冯西桥.核反应堆压力容器的LBB分析[R].北京:清华大学核能技术设计研究院,1997.

[10]谢希德.创造学习的思路[N].人民报,1998-12-25(10).

[11]王明亮.中国学术期刊系统工程[EB/OL].北京:中国书籍出版社,1998-08-16[1998-09-26].http://www.ie.nthu.edu.tw/info/ie.newie.htm(Big5).

\subsection{附录(必要时)}
附录中主要列入正文内不便列出的过分冗长的公式推导,供查读方便所需的辅助性数学工具或表格、重复性数据图表、计算程序及说明等。附录依次为附录1,附录2……等,“附录X”三字居中(小三号黑体)。附录中的图表公式另编排序号,与正文分开。
\subsection{致谢}
作者对给予指导、各类资助和协助完成研究工作以及提供各种对论文工作有利条件的单位及个人表示感谢。“致谢”二字居中(小三号黑体)。
\subsection{作者简介}
内容一般包括:姓名、性别、出生日期、籍贯、最后学历(学位)、毕业院校、工作经历;在学期间参加的研究项目、发表论文、申请专利、获奖情况等。

分类按时间顺序列出作者在攻读学位期间取得的与学位论文相关的研究成果,含参加的研究项目、获奖情况、专利、专著、发表学术论文(或有正式录用函等),著作及学术论文等的书写格式要求与参考文献相同。“作者简介”字体居中(小三号黑体)。
\subsection{导师简介}
内容一般包括:姓名、职称、职务、主要社会兼职(如省或全国学术团体理事以上职务)、主要研究领域、学术经历、主要研究成果。该简介由导师自撰或经导师签名认可。导师简介字数为300-500左右。“导师简介”字体居中(小三号黑体)。
\section{页眉和页码要求}
页眉从正文开始至全文结束,在版芯上边线隔一行1.5磅加粗、细双线(粗线在上),其上居中打印页眉,字体、字号与正文同相同采用小四号宋体,奇数页写“广东海洋大学博(或硕)士学位论文”,偶数页写论文题目。

页码从正文开始至全文结束按阿拉伯数字连续编排,前置部分(中文摘要、英文摘要、目录、符号说明等)用罗马数字分别单独编排。页码位于页面底端,居中书写。

\section{量和单位规范要求}
严格执行《量和单位》国家标准(GB3100-93、GB3101-93、GB3102.1-13-93等共15项)有关量和单位的规定(具体要求请参阅《常用量和单位》,计量出版社,1996)。单位的名称书写可以采用国际通用符号,也可以用中文名称,但全文应统一,不要两种混用。
